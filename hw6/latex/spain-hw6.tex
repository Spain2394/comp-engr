%%%%%%%%%%%%%%%%%%%%%%%%%%%%%%%%%%%%%%%%%
% Lachaise Assignment
% LaTeX Template
% Version 1.0 (26/6/2018)
%
% This template originates from:
% http://www.LaTeXTemplates.com
%
% Authors:
% Marion Lachaise & François Févotte
% Vel (vel@LaTeXTemplates.com)
%
% License:
% CC BY-NC-SA 3.0 (http://creativecommons.org/licenses/by-nc-sa/3.0/)
%
%%%%%%%%%%%%%%%%%%%%%%%%%%%%%%%%%%%%%%%%%

%----------------------------------------------------------------------------------------
%	PACKAGES AND OTHER DOCUMENT CONFIGURATIONS
%----------------------------------------------------------------------------------------

\documentclass{article}

\input{structure.tex} % Include the file specifying the document structure and custom commands
\usepackage{graphicx}
\newcommand{\ts}{\textsuperscript}
<<<<<<< HEAD
\usepackage{float}

=======
>>>>>>> 31adf8bead0ca3f0783d74f4a70004e23d603956

%----------------------------------------------------------------------------------------
%	ASSIGNMENT INFORMATION
%----------------------------------------------------------------------------------------

\title{Computational Engineering - Engr 8103 \\ Problem Set \#6} % Title of the assignment

\author{Allen Spain\\ \texttt{avs81684@uga.edu}} % Author name and email address

\date{University of Georgia --- 22 October 2019 } % University, school and/or department name(s) and a date

%----------------------------------------------------------------------------------------

\begin{document}

\maketitle % Print the title

%----------------------------------------------------------------------------------------
%	INTRODUCTION
%----------------------------------------------------------------------------------------

\section*{Problem 1} % Unnumbered section
Show that the following is another valid second order Runge-Kutta method:

\noindent
From the 2\ts{nd} order Taylor series method:

\noindent
<<<<<<< HEAD
Where:
=======
Notation:
>>>>>>> 31adf8bead0ca3f0783d74f4a70004e23d603956

\begin{align*}
	x  \rightarrow x(t) \\
	f  \rightarrow f(t,x) \\
	f_{t}  \rightarrow \frac{\partial f}{\partial t} \\
	f_{x}  \rightarrow \frac{\partial f}{\partial x}
\end{align*}
\begin{align*}
	\boxed{x(t + h) = x + hf + \frac{h^{2}}{2}(f_{t} + ff_{x})}
\end{align*}

\noindent
From the 2\ts{nd} order Runge-Kutta method:

\begin{align*}
	K_{1} = hf \\
	K_{2} = hf(t + \frac{2}{3}h, x + \dfrac{2}{3}K_{1}) = h(f +  \frac{2}{3}h(f_{t} + ff_{x})) \\
	x(t + h) = x(t) + (K_{1} + 3K_{2})/4 \\
	= x(t) + \dfrac{hf(t,x)}{4} + 3/4(h(f + \frac{2}{3}h(f_{t} + ff_{x}))) \\
	= x(t) + \dfrac{hf}{4} + \dfrac{3hf}{4} + \dfrac{h^{2}}{2}(f_{t} + ff_{x}) \\
	\boxed{x(t + h) = x(t) + hf + \frac{h^{2}}{2}(f_{t} + ff_{x})}
 \end{align*}

<<<<<<< HEAD
\section*{Problem 2}
Show that when the fourth-order Runge-Kutta method is applied to the problem $ x^{\prime} = 2x $, the formula for advancing this solution will be

\begin{align*}
	x(t + h) = \big[1 + 2h + 2h^{2} + \frac{4}{3}h^{3} + \frac{2}{3}h^{4} \big] x(t) \\
\end{align*}

From the 4\ts{th} order Runge-Kutta method:
\begin{align*}
	K_{1} = hf = 2xh \\
	K_{2} = hf(t + \frac{h}{2}, x + \frac{K_{1}}{2}) = 2xh + 2h^{2}x \\
	K_{3} = hf(t + \frac{h}{2}, x + \frac{K_{2}}{2}) = 2xh + 2xh^{2} + 2xh^{3} \\
	K_{4} = hf(t + \frac{h}{2}, x + \frac{K_{3}}{2}) = 2xh + 4xh^{2} + 4xh^{3} + 4xh^{4} \\
	x(t + h) = x(t) + (K_{1} + 2K_{2} + 2K_{3} + K_{4})/6 \\
	= x(t) + (2xh + 2(2xh + 2xh^{2}) \\ + 2(2xh + 2xh^{2} + 2xh^{3}) + (2xh + 4xh^{2} + 4xh^{3} + 4xh^{4}))/6 \\
	x(t + h) = x(t)[1 + 2h + 2h^{2} + \frac{4}{3}h^{3} + \frac{2}{3}h^{4}]
\end{align*}

\section*{Problem 3}
Consider the following ODE:
\begin{align*}
	x^{\prime} = -y \quad x(0) = 1
	y^{\prime} = x \quad y(0) = 0
\end{align*}

\noindent
(a) Write a Matlab code that solves this equation system on the interval [0, 10] using
- Second order Taylor series method
- Second order Runge-Kutta method (does not matter which one you use) - Fourth order Runge-Kutta method

\lstinputlisting{/Users/allenspain/Documents/Development/MATLAB/comp-engr/hw6/ODEComparison.m}

(b) Run your code for h = 0.1, 0.25, 0.5 and 1. Include a hard copy of all four graphs along with your HW solutions.

\begin{figure}[H]
  \includegraphics[width=\linewidth]{docs/h1.jpg}
  \caption{h = 1}
  % \label{fig:boat1}
\end{figure}

\begin{figure}[H]
  \includegraphics[width=\linewidth]{docs/05.jpg}
  \caption{h = 0.5}
\end{figure}

\begin{figure}[H]
  \includegraphics[width=\linewidth]{docs/h25.jpg}
  \caption{h = 0.25}
\end{figure}

\begin{figure}[H]
  \includegraphics[width=\linewidth]{docs/0point1.jpg}
  \caption{h = 0.1}
\end{figure}


\section*{Problem 4}
Consider the following IVP

\begin{align*}
	x^{\prime} = x(1-x) \\
	x(0) = 0.01
\end{align*}
with the solution function $ x(t) = 1 - 1/(1 + e^{t}/99)$

\noindent
(a) Modify adaptive.m available on the course website to implement the Runge-Kutta-Fehlberg algorithm, provided below, to solve this IVP for $0 \leq t \leq 16$ The code should plot the analytical solution and the numerical solution on the same figure.

\lstinputlisting{/Users/allenspain/Documents/Development/MATLAB/comp-engr/hw6/RKF.m}

\begin{figure}[H]
  \includegraphics[width=\linewidth]{docs/01-rfk.jpg}
  \caption{tolerance = 0.01}
\end{figure}


\begin{figure}[H]
  \includegraphics[width=\linewidth]{docs/001-rfk.jpg}
  \caption{tolerance = 0.001}
\end{figure}


=======
>>>>>>> 31adf8bead0ca3f0783d74f4a70004e23d603956
\end{document}
