%%%%%%%%%%%%%%%%%%%%%%%%%%%%%%%%%%%%%%%%%
% Lachaise Assignment
% LaTeX Template
% Version 1.0 (26/6/2018)
%
% This template originates from:
% http://www.LaTeXTemplates.com
%
% Authors:
% Marion Lachaise & François Févotte
% Vel (vel@LaTeXTemplates.com)
%
% License:
% CC BY-NC-SA 3.0 (http://creativecommons.org/licenses/by-nc-sa/3.0/)
%
%%%%%%%%%%%%%%%%%%%%%%%%%%%%%%%%%%%%%%%%%

%----------------------------------------------------------------------------------------
%	PACKAGES AND OTHER DOCUMENT CONFIGURATIONS
%----------------------------------------------------------------------------------------

\documentclass{article}

\input{structure.tex} % Include the file specifying the document structure and custom commands
\usepackage{graphicx}


%----------------------------------------------------------------------------------------
%	ASSIGNMENT INFORMATION
%----------------------------------------------------------------------------------------

\title{ENGR 8103: Problem Set \#5} % Title of the assignment

\author{Allen Spain\\ \texttt{avs81684@uga.edu}} % Author name and email address

\date{University of Georgia --- 10 October 2019 } % University, school and/or department name(s) and a date

%----------------------------------------------------------------------------------------

\begin{document}

\maketitle % Print the title

%----------------------------------------------------------------------------------------
%	INTRODUCTION
%----------------------------------------------------------------------------------------


\section*{Problem 1} % Unnumbered section
If $T = A(h) + a_{1} h^{1/2} + a_{2} h^{2/2} + a_{3} h^{3/2},$ then what combination of $A(h)$ and $A(h/2)$ should give an accurate estimate of $T$ ?

\begin{equation}
T = A(h) + a_{1}h^{1/2} + a_{2}h^{2/2} + a_{3}h^{3/2}  + \cdots\\
\end{equation}

\begin{equation}
T = A(h/2) + \frac{a_{1}h^{1/2}}{2^{1/2}} + \frac{a_{2}h^{1}}{2} +\\ \frac{a_{3}h^{3/2}}{2^{3/2}} + \cdots
\end{equation}

Combine (1) and equation (2) as follows $T - \frac{-T}{\sqrt{2}} $ this yields the resultant equation:

\begin{equation}
T - \frac{T}{\sqrt{2}} =  A(h/2) - \frac{A(h)}{\sqrt{2}} + \underbrace{ \frac{a_{2}h^{1}}{2} + \frac{a_{3}h^{3/2}}{2^3/2} - \frac{a_{3}h^{3/2}}{\sqrt{2}} + \cdots}_\text{Error (E)}
\end{equation}

Therefore the combination of $A(h) and A(h/2)$ is as follows:

\begin{equation}
T = \frac{A(h/2) - \frac{A(h)}{\sqrt{2}}}{1-\frac{1}{\sqrt{2}}} + E
\end{equation}



\section*{Problem 2} % Unnumbered section
Consider a second order approximation $A(h)$ to $T$ such that
\begin{equation}
T = A(h) + a_{2}h^{2} + a_{4}h^{4} + a_{6}h^{6} + \cdots
\end{equation}

a) Find a sixth order approximation to $T$ using $A(h), A(2h), A(3h)$

Using step size $ 2h $

\begin{equation}
-T = A(2h) + 4a_{2}h^{2} + 16a_4h^{4} + 32a_{6}h^{6} + \cdots
\end{equation}

Combining equation 5 and 6.
\begin{equation}
-T = A(2h) + 4a_{2}h^{2} + 16a_4h^{4} + 32a_{6}h^{6} + \cdots \\
\end{equation}

\begin{equation}
	4T = 4A(h) + 4a_{2}h^{2} + 4a_{4}h^{4} + 4a_{6}h^{6} + \cdots \\
\end{equation}

\begin{equation}
	 \approx 3T = 4A(h) - A(2h) - 12a_{4}h^{4} - \frac{60a_{6}}{h^{6}}{3} \cdots \\
\end{equation}

\begin{equation}
  T = \dfrac{4A(h) - A(2h)}{3} - 4a_{4}h^{4} - 20a_{6}h^{6} + \cdots
\end{equation}

Now coarse correction using step size $3h$

\begin{equation}
  T = A(3h) + 9a_{2}h^{2} + 81a_{4}h^{4} + 729a_{6}h^{6} + \cdots
\end{equation}

Combining equation 12 and 5
\begin{equation}
 9T = 9A(h) + 9a_{2}h^{2} + 9a_{4}h^{4} + 9a_{6}h^{6} + \cdots
\end{equation}

\begin{equation}
-T = A(3h) + 9a_{2}h^{2} + 81a_{4}h^{4} + 729a_{6}h^{6} + \cdots
\end{equation}

\begin{equation}
\approx 9T - T = 9A(h) - A(3h) - 72a_{4}h^{4} - 720a_{6}h^{6} + \cdots
\end{equation}

This yields...
\begin{equation}
T = \frac{9A(h) - A(3h)}{8} - 9a_{4}h^{4} - 90a_{6}h^{6} + \cdots
\end{equation}

Cancel out high order term using equation 8 and equation 15
\begin{equation}
(4T = 4A(h) + 4a_{2}h^{2} + 4a_{4}h^{4} + 4a_{6}h^{6} + \cdots) * 9
\end{equation}

\begin{equation}
(T = \frac{9A(h) - A(3h)}{8} - 9a_{4}h^{4} - 90a_{6}h^{6} + \cdots) * -4
\end{equation}

Subtract the equations
\begin{equation}
5T = \frac{3A(h)}{2} - \frac{3A(2h)}{5} - \frac{A(3h)}{10} + 180a_{6}h^{6}
\end{equation}

Solving gives you a 6th approximation of of $T$
\begin{equation}
T = \frac{3A(h)}{2} - \frac{3A(2h)}{5} - \frac{A(3h)}{10} + 180a_{6}h^{6}
\end{equation}

b) Applying part a)
\begin{equation}
T = \underbrace{\frac{3A(h)}{2} - \frac{3A(2h)}{5} - \frac{A(3h)}{10} + 180a_{6}h^{6} + \cdots}_\text{A(h)}
\end{equation}


\begin{equation}
A(h) = \frac{1}{h^2}[f(x+h) -2f(x) + f(x-h)] + \mathcal{O}(h^{6})
\end{equation}

By finding the common denominator of all of the fractional componenents in equation 20 you can solve the following
\begin{equation}
f''(x) = \frac{1}{36h^{2}}[ 49f(x+h)+ 49f(x+2h) + 49 f(x + 3h) - 294f(x) + 49f(x - h) + 49f(x -2h) + 49f(x -3h)] + \mathcal{O}(h^{6})
\end{equation}


\section*{Problem 3} % Unnumbered section

\noindent
(a): Consider the following IVP ?
% Math equation/formula
\begin{equation}
	x ^{\prime} = 1 + t + x^2
\end{equation}

\begin{equation}
  x(0) = 1
\end{equation}

\lstinputlisting{/Users/allenspain/Documents/Development/MATLAB/comp-engr/hw5/euler.m}

\begin{verbatim}
-------------------

Ouput:

index:0 1
index:1 3
index:2 14
index:3 213
index:4 45586

-------------------
\end{verbatim}

\begin{figure}
  \includegraphics[width=\linewidth]{docs/euler.jpg}
  \caption{Euler's method for approximating $x$}
  \label{fig:boat1}
\end{figure}

From Euler's method $ x(2) = 14$ \\


(b): Use second-order Taylor Series method to estimate $x(2)$ with step size $h = 1$

\lstinputlisting{/Users/allenspain/Documents/Development/MATLAB/comp-engr/hw5/taylor_series.m}

\begin{verbatim}
-------------------
Ouput:

index:0 1
index:1 5.500000e+00
index:2 2.156250e+02
index:3 1.007266e+07
index:4 1.021957e+21

-------------------
\end{verbatim}

\noindent
Using Taylor series second order approximation $x(2) \approx 215$\\


\noindent

(c): Use third-order Taylor Series method to estimate $x(2)$ with step size  $ h = 1$ \\
$ x(2) \approx  399410 $

\lstinputlisting{/Users/allenspain/Documents/Development/MATLAB/comp-engr/hw5/taylor_series3.m}


\begin{verbatim}
-------------------
Ouput:

index:0 1
index:1 3
index:2 14
index:3 213
index:4 45586

-------------------
\end{verbatim}



\end{document}
