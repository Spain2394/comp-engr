%%%%%%%%%%%%%%%%%%%%%%%%%%%%%%%%%%%%%%%%%
% Lachaise Assignment
% LaTeX Template
% Version 1.0 (26/6/2018)
%
% This template originates from:
% http://www.LaTeXTemplates.com
%
% Authors:
% Marion Lachaise & François Févotte
% Vel (vel@LaTeXTemplates.com)
%
% License:
% CC BY-NC-SA 3.0 (http://creativecommons.org/licenses/by-nc-sa/3.0/)
%
%%%%%%%%%%%%%%%%%%%%%%%%%%%%%%%%%%%%%%%%%

%----------------------------------------------------------------------------------------
%	PACKAGES AND OTHER DOCUMENT CONFIGURATIONS
%----------------------------------------------------------------------------------------

\documentclass{article}

\input{structure.tex} % Include the file specifying the document structure and custom commands
\usepackage{graphicx}
\newcommand{\ts}{\textsuperscript}
\usepackage{float}


%----------------------------------------------------------------------------------------
%	ASSIGNMENT INFORMATION
%----------------------------------------------------------------------------------------

\title{Computational Engineering - Engr 8103 \\ Problem Set \#4} % Title of the assignment

\author{Allen Spain\\ \texttt{avs81684@uga.edu}} % Author name and email address

\date{University of Georgia --- 12 December 2019 } % University, school and/or department name(s) and a date

%----------------------------------------------------------------------------------------

\begin{document}

\maketitle % Print the title

%----------------------------------------------------------------------------------------
%	INTRODUCTION
%----------------------------------------------------------------------------------------

\section*{Question 1} % Unnumbered section
(10 pts.) Derive the three point Gaussian quadrature formula. In other words, find coefficients A that make the following approximation exact for polynomials up to fifth degree.

\begin{gather*}
  \int_{-1}^{1} dx \simeq Af(-\alpha) + Bf(0) + Cf(\alpha)
\end{gather*}

Express its general form
\begin{gather*}
  \int_{a}^{b} dx \approx
\end{gather*}

General form for Gaussian quadrature is

\begin{gather*}
  \int_{a}^{b} dx \approx \sum_{i=1}^{n} w_{i} f(x_{i})
\end{gather*}

Now you scale and translate the interval
\begin{gather*}
\int_{a}^{b} f(x)dx = \frac{b-a}{2}\int_{-1}^{1}f \left( \frac{b-a}{2}x + \frac{a+b}{2} \right) dx
\end{gather*}

Converting to the discretized formula
\begin{gather*}
\int_{a}^{b} f(x)dx \approx \frac{b-a}{2} \sum_{i=1}^{n}w_{i}f \left(  \frac{b-a}{2}x_{i} + \frac{a+b}{2}      \right)
\end{gather*}

Expanding terms
\begin{gather*}
  \int^{b}_{a} f(x) dx \approx = \frac{(b-a)}{2} \left( Af \left( \frac{(b-a)}{2} (-\alpha) + \frac{(a+b)}{2} \right) + Bf \left( \frac{(b-a)}{2})(0) + \frac{(a+b)}{2} \right) + Cf \left( \frac{(b-a)}{2}(\alpha) + \frac{(a+b)}{2}  \right)      \right)
\end{gather*}

After simplifying, you can generate a system of equations, this is done for all simple x functions that increase in degree up to $f(x) = x^{5}$

\begin{gather*}
  f(x) = 1, \quad \int_{-1}^{1}dx = A + B + C = 2 \\
  f(x) = x, \quad \int_{-1}^{1}xdx = C \alpha - A \alpha = 0 \\
  f(x) = x^{2}, \quad \int_{-1}^{1}x^{2}dx =  [\frac{x^{3}{3}}]^{1}_{-1} = A\alpha^{2} = C\alpha^{2} = \frac{2}{3}\\
  f(x) = x^{3}, \quad \int_{-1}^{1}x^{3}dx =  \frac{x^{4}}{4}]^{1}_{-1} = C\alpha^{3} + A\alpha^{3} = 0\\
  f(x) = x^{4},\quad \int_{1}^{-1}x^{4}dx = [\frac{x^{5}}{5}]^{1}_{-1} = A\alpha^{4} + C\alpha^{4} = \frac{2}{5}\\
  f(x) = x^{5}, \quad \int^{1}_{-1}x^{5}dx = [\frac{x^{6}}{6}]^{1}_{-1} = C\alpha^{5} - A\alpha^{5} = 0\\
\end{gather*}

Now you can collect terms

\begin{gather*}
    C\alpha = A\alpha \rightarrow C = A\\
    A\alpha^{2}  + C\alpha^{2} = \frac{2}{3} \rightarrow A\alpha^{2} + (A)\alpha^{2} = \frac{2}{3} \rightarrow A = \frac{1}{3\alpha^{2}}\\
    \text{therefore } A  = \frac{1}{3\alpha^{2}} = C\\
    A\alpha^{4} + C\alpha^{4}= \frac{2}{5} \rightarrow \frac{1}{3a^{2}}\alpha^{4} = \frac{2}{5} \rightarrow \frac{\alpha^{2}}{3} + \frac{\alpha^{2}}{3} = \frac{2}{5}
    A\alpha^{2} + C\alpha^{2} = \frac{2}{3}\\
    \rightarrow \frac{2\alpha^{2}}{3} = \frac{ 2}{5} \rightarrow 10 \alpha^{2} = 6 \rightarrow \alpha = \sqrt{\frac{3}{5}} \\
    A = 3\alpha^{2} \rightarrow A = \frac{1}{3\frac{3}{5}}^{2} \alpha \rightarrow A = \frac{5}{9}\\
    \text{subsititute}
    A = C = \frac{5}{9}\\
    A + B + C = 2 \rightarrow \frac{5}{9} + B + (\frac{5}{9}) = 2 \\ \therefore B = \frac{8}{9}
\end{gather*}

Substituting back into the original expanded sum:

\begin{gather*}
  \int_{a}^{b} \approx \frac{(b-a)}{2} (\frac{5}{9}) f(\frac{a+b}{2} - \sqrt{\frac{3}{5}} \frac{(b-a)}{2}) + \frac{8}{9}f(\frac{(a+b)}{2}) + \frac{5}{9} f(\sqrt{\frac{3}{5}}\frac{(b-a)}{2} + \frac{a+b}{2}))
\end{gather*}

\section*{Question 2} % Unnumbered section
(10 pts.) Derive a custom numerical integration formula of the form

\begin{gather*}
  \int_{0}^{pi} = f(x)dx \approx Af(0)  + Bf(\frac{\pi}{2}) + Cf(\pi)
\end{gather*}

that is exact for all functions of type: $ f (x) = a + b sinx + c cosx$

\begin{gather*}
  \int_{0}^{\pi} a + bsin(x)+ ccos(x) = [ax]^{\pi}_{0} - [bcos(x)]_{0}^{\pi} + [csin(x)]_{0}^{\pi}\\
  \therefore a\pi - b(-1-1) + c(0-0) = a\pi + 2b \approx Af(0) + Bf(\frac{\pi}{2}) + Cf(\pi)\\
\end{gather*}

Subsituting back into the
\begin{gather*}
A(a + bsin((\theta)) + ccos((\theta))) + B(a + bsin((\pi)) + ccos((\pi))) + C(a + bsin((\pi)) + ccos((\pi))) \approx a\pi + 2b
\end{gather*}
simplifying
\begin{gather*}
% A(a + c) + B(a + b) + C(a − c) \approx a \pi + 2b
A(a+c) + B(a+b) + C(a-c) \approx a\pi +2b
\end{gather*}

\noindent
Notice $c(A-C) = 0 \Rightarrow A = C$\\
Also notice $bB = 2b, so B = 2 $\\
Then, because $a(A + B + C) = \alpha \pi and A = C, A = C = \frac{\pi - 2}{2}$

Finally, no change in variable is needed because the question explicitly asks for the interval of $[0,\pi]$, the formula is:

\begin{gather*}
  \int_{0}^{\pi} \approx \frac{\pi-2}{2}f(0) + 2f(\frac{\pi}{2}) + \frac{\pi - 2}{2}f(\pi)
\end{gather*}


\section*{Question 3} % Unnumbered section
(15 pts.) Consider the following integral

\begin{gather*}
\int_{0}^{\pi} (3 + e^{\frac{-x}{2}} - sin(x) + 2cos(x))dx
\end{gather*}

\noindent
(a)
i.
\begin{gather*}
  f(x_{0})= 3 + e^{\frac{0}{2}} - sin((0)) + 2cos((0)) = 6\\
  2f(x_{1}) = 2(3 + e^{-\frac{\pi}{2}}) - sin(\frac{\pi}{2}) + 2cos(\frac{\pi}{2})  = 2e^{-\frac{\pi}{4}} + 4\\
  f(x_{2}) = 3 + e^{-\frac{\pi}{2}} - sin(\pi) + 2cos(\pi) = e^{\frac{-\pi}{2}} + 1
\end{gather*}
So the integral estimation is $9.51883$

ii. (2 pts.) Simpson’s method using 3 function evaluations at $(0,\frac{\pi}{2},\pi)$ Simpsons can be written
as:

\begin{gather*}
  \int_{b}^{a}f(x)dx \approx \frac{b-a}{6}\left(       f(a) + 4f\frac{(a+b)}{2}  + f(b)\right)
\end{gather*}

This matches our size and bounds so

\begin{gather*}
  \frac{\pi}{6}\int_{0}^{\pi}f(x)dx \approx (f(0) + 4f(\frac{\pi}{2}) +  f(\pi))
\end{gather*}

substitute in for $f(x)$:

\begin{gather*}
f(0) = (3 + e^{\frac{-0}{2}} - sin(0) + 2cos(0)) = 6 \\
4f(\frac{\pi}{2}) = 4(3 + e^{-(\pi/2)/2} - sin(\pi/2) + 2cos(\pi/2)) = 9.82375... \\
f(\pi) = (3 + e^{-(\pi)/2} - sin(\pi) + 2cos(\pi)) = 1.20787
\end{gather*}

So the integral estimate is $ \approx 8.91774$


iii. (2 pts.) Gaussian quadrature using 2 function evaluations The formula is:

\begin{gather*}
  \int_{-1}^{1}f(x)dx = Af(-\alpha) + Bf(\alpha)
\end{gather*}

\begin{gather*}
  f(x) = 1, \quad \int_{-1}^{1}dx = 2 = A+B\\
  f(x) = x, \quad \int_{-1}^{1}xdx = 0 = B\alpha - A\alpha\\
  f(x) = x^{2}, \quad \int_{-1}^{1} x^{2}dx = \frac{2}{3} = A\alpha^{2} + B\alpha^{2}
\end{gather*}

\noindent
Notice, $A = B$ again, so $A = B = 1 \therefore A+B = 2$
Then substitute $A\alpha^{2}  + B\alpha^{2} = \frac{2}{3}$ so $
\alpha = \sqrt{\frac{1}{3}}$ \\
Substituting

\begin{gather*}
  \int_{0}^{\pi}f(x)dx \approx \frac{\pi}{2}((1)f(\frac{\pi}{2}    - \sqrt{\frac{1}{3}}    \frac{\pi}{2}) + (1)f(\sqrt{\frac{1}{3}}\frac{\pi}{2}  + \frac{\pi}{2}))) \Rightarrow \text{plug and chug} \Rightarrow 9.07112...
\end{gather*}

iv.(2 pts.) Gaussian quadrature using 3 function evaluations (Problem 1).

Now, $a = 0$ and $ b = \pi$

\begin{gather*}
 \int_{0}^{\pi} f(x)dx \approx \frac{\pi}{2} \left(\frac{5}{9} f(\frac{a+b}{2} - \sqrt{\frac{3}{5}} \frac{b-a}{2}           )  + \frac{8}{9}f(\frac{\pi}{2}       )                  + f(\sqrt{\frac{3}{5}}   \frac{\pi}{2} + \frac{\pi}{2} )   \right)
%
\end{gather*}

Then, pluggin in known values into $f(x)$
\begin{gather*}
 \approx \pi (2.98165 + 2.18305 + 0.56971) \approx 9.00759...
\end{gather*}



% part v
v. (2 pts.) Gaussian quadrature using 4 function evaluations

\begin{gather*}
\int_{-1}^{1} f(x)dx \approx \frac{18 + \sqrt{30}}{36} f(\sqrt{\frac{2}{7} + \frac{2}{7} \sqrt{\frac{6}{5}}}) \\ + \frac{18 + \sqrt{30}}{36} f(\sqrt{\frac{2}{7} - \frac{2}{7} \sqrt{\frac{6}{5}}}) \\  + \frac{18 + \sqrt{30}}{36} f(\sqrt{\frac{2}{7} + \frac{2}{7}  \sqrt{\frac{6}{5}}}) \\ + \frac{18 + \sqrt{30}}{36} f(\sqrt{\frac{2}{7} - \frac{2}{7} \sqrt{\frac{6}{5}}})
%$
\end{gather*}

Adjusting for change of variable
\begin{gather*}
  \int_{a}^{b} f(x)dx \approx \frac{b-a}{2} \frac{18 + \sqrt{30}}{36} f(\sqrt{\frac{2}{7} + \frac{2}{7} \sqrt{\frac{6}{5}}}  \frac{b-a}{2} +     ) \\ \frac{b-a}{2} \frac{18 + \sqrt{30}}{36} f( \frac{b-a}{2} - \frac{b-a}{2}  \sqrt{\frac{2}{7} - \frac{2}{7} \sqrt{\frac{6}{5}}} ) \\  +\frac{b-a}{2} \frac{18 + \sqrt{30}}{36} f ( \frac{b-a}{2} + \frac{b-a}{2}\sqrt{\frac{2}{7} + \frac{2}{7}  \sqrt{\frac{6}{5}}} ) \\ + \frac{b-a}{2} \frac{18 + \sqrt{30}}{36} f(\sqrt{\frac{2}{7} - \frac{2}{7} \sqrt{\frac{6}{5}}} \frac{b-a}{2} + \frac{b-a}{2} )
\end{gather*}

Subsituting the interval back in

\begin{gather*}
  \approx \frac{\pi}{2} (1.96446 + 2.44734 + 0.95885 + 0.37089) \approx 9.01878
\end{gather*}


\noindent
% This question doesn't end part 6
vi. (2 pts.) Method you derived in Problem 2.
\begin{gather*}
  \int_{0}^{\pi}f(x)dx \approx \frac{\pi -2}{2}f(0) + 2f(\frac{\pi}{2}) + \frac{\pi - 2}{2}f(\pi)
\end{gather*}


\noindent
(b) (3 pts.) Take this interal and manually find its exact value. Find the absolute errors for each of the six methods above.

\begin{gather*}
  \int_{0}^{3}(3 + e^{\frac{-x}{2}} - sinx + 2cosx) dx
  = \int_{0}^{\pi} 3dx + \int_{0}^{\pi}e^{\frac{-x}{2}}dx + \int_{0}^{\pi}sin(x)dx + \int_{0}^{\pi}2cosxdx\\
  = [3x]^{\pi}_{0} + [-2e^{\frac{-x}{2}}]_{0}^{\pi} + [cos(x)]^{\pi}_{0} + [2sin(x)]^{\pi}_{0} = 3\pi + (-2e^{\frac{-\pi}{2}} + 2) - (1+1) + (0-1) \approx 9.00901
\end{gather*}

\noindent
The errors for the methods are in order as follows:\\
i.  $|9.00901 - 9.51883| \approx 0.50982$ \\
ii. $|9.00901 - 8.91774| \approx 0.09127$ \\
iii.$|9.00901 - 9.07112| \approx 0.06211$ \\
iv. $|9.00901 - 9.00759| \approx 0.00142$ \\
v. $|9.00901 - 9.01878| \approx 0.00977$ \\
vi. $|9.00901 - 9.02609| \approx 0.0170$

\end{document}
